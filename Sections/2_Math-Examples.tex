\section{Math Examples}

This section demonstrates various mathematical features supported by the template. Every package listed under the \enquote{Mathematics} category in \texttt{packages.sty} is represented here with a usage example. Some of these examples are purely illustrative and are not necessarily mathematically meaningful.

The \gls{gcd} and \gls{lcm} are fundamental concepts in number theory. The \gls{lcm} of two numbers is the smallest number that is a multiple of both.
\begin{equation*}
    \gcd(a, b) = \mathit{\glsxtrshort{gcd}}(a, b)
\end{equation*}
\begin{equation*}
    \lcm(a, b) = \mathit{\glsxtrshort{lcm}}(a, b)
\end{equation*}

We begin with examples of custom\-/defined relation symbols:
\begin{equation*}
    x \defeq y + z
\end{equation*}
\begin{equation*}
    a \whateq b
\end{equation*}
\begin{equation*}
    p \hateq q
\end{equation*}
\begin{equation*}
    f \coleq g
\end{equation*}
\begin{equation*}
    M \meq N
\end{equation*}
\begin{equation*}
    U \qmeq V\transpose
\end{equation*}

The number of permutations of $r$ items from a set of $n$ elements is given by:
\begin{equation*}
    \perm{n}{r} = \frac{n!}{(n - r)!}
\end{equation*}

The number of combinations (binomial coefficient) is:
\begin{equation*}
    \choose{n}{r} = \comb{n}{r} = \frac{n!}{r!(n - r)!}
\end{equation*}

We can also use floor and ceiling functions:
\begin{equation*}
    \ceil*{1.2} = 2 \qquad \floor*{2.2} = 2
\end{equation*}

Expected value notation:
\begin{equation*}
    \expect{X} = \sum_{x \in \mathcal{X}} x \cdot P(X = x)
\end{equation*}

Variance:
\begin{equation*}
    \Var(X) = \sigma^2(X) = \expect{(X - \expect{X})^2} = \sum_{x \in \mathcal{X}} (x - \expect{X})^2 \cdot P(X = x)
\end{equation*}

Some commonly used mathematical sets and transforms:
\begin{equation*}
    \gls{N} \subset \gls{Z} \subset \gls{Q} \subset \gls{R} \subset \gls{C}, \quad \laplace, \quad \compl{A}
\end{equation*}

Equations, partially to demonstrate \texttt{glossaries\-/extra}:
\begin{equation*}
    E = m\gls{c}^2
\end{equation*}

A piecewise\-/defined function can be written as:
\begin{equation*}
    f(x) = 
    \begin{cases}
        x^2 & \cif{x \geq 0} \\
        -x  & \cotherwise
    \end{cases}
\end{equation*}

We can use symbols such as $\forall$, $\exists$, $\Rightarrow$, $\nexists$, and $\neg$ for logic notation.

We also use \texttt{mathclap} to fix spacing issues:
\begin{equation*}
    \sum_{1 \leq i < j \leq n} a_{ij}
\end{equation*}
\begin{equation*}
    \sum_{\mathclap{1 \leq i < j \leq n}} a_{ij}
\end{equation*}

\begin{theorem}[Intervals]
    Let $f$ be a continuous function in a closed interval $[a,b]$. Then $f$ is bounded and attains its bounds on $[a,b]$.
\end{theorem}

\begin{proof}
    By the Extreme Value Theorem, a continuous function on a closed interval attains its maximum and minimum.
\end{proof}

Bold vector notation using \texttt{bm}:
\begin{equation*}
    \bm{F} = m \bm{a}
\end{equation*}

We can cancel out terms to simplify expressions:
\begin{equation*}
    \frac{\cancel{a} \cdot b}{\cancel{a}} = b
\end{equation*}

The \texttt{nicefrac} package provides better inline fractions to write text: $\nicefrac{a}{b}$ is cleaner than $\frac{a}{b}$ or $a/b$.
