\section{Code Examples}

This section demonstrates how to include source code using the \texttt{minted} package, which provides syntax highlighting for a wide range of languages. Code is typically presented in a \texttt{listing} environment and can be customised with captions and labels for cross\-/referencing. The \texttt{minted} package also supports inputting external files and allows for inline maths, though care must be taken when combining code and mathematical symbols. Furthermore, the \texttt{simplebox} command can be used to highlight key blocks, notes, or particularly important code for readability and emphasis.

The use of \gls{recursion} in this algorithm makes it efficient in solving \gls{fibonacci}. Its time \gls{complexity} is $\bigO{n}$.

\begin{listing}[H]
    \begin{minted}{python}
from collections.abc import Generator

# This is an example. The next line is 67 (65 + 2) characters long:
# 12345678901234567890123456789012345678901234567890123456789012345
class Math:
    @staticmethod
    def fib(n: int) -> Generator[int, None, None]:
        """Fibonacci series up to n."""
        a, b = 0, 1

        while a < n:
            yield a
            a, b = b, a + b

result = sum(Math.fib(42))
print(f"The answer is {result}")
    \end{minted}
    \caption{Python implementation of the Fibonacci sequence.}
    \label{lst:fibonacci}
\end{listing}

External files can also be included, making it easy to manage larger projects:

\begin{listing}[H]
    \inputminted{Matlab}{Code/HelloWorld.m}
    \caption{A simple MATLAB programme.}
    \label{lst:matlabHelloWorld}
\end{listing}

Mathematical expressions can be embedded in code blocks, but should be commented on properly or placed in strings to avoid confusion with the syntax:

\begin{listing}[H]
    \begin{minted}{python}
test
$A \wedge B$
test
# $A \wedge B$
test
|$A \wedge B$|
test
    \end{minted}
    \caption{Using math inside \texttt{minted} environments.}
    \label{lst:mathMinted}
\end{listing}

The following example shows how one might embed \LaTeX\-/style maths within comments and strings in C\#:

\begin{listing}[H]
    \begin{minted}{csharp}
string title = "This is a Unicode $\pi$ in the sky"
/**
 * Defined as $\pi=\lim_{n\to\infty}\frac{P_n}{d}$ where $P$ is the perimeter of an $n$-sided regular polygon circumscribing a circle of diameter $d$.
 */
const double pi = 3.1415926535
    \end{minted}
    \caption{Using mathematical notation in C\# code comments and strings.}
    \label{lst:csharpMath}
\end{listing}
